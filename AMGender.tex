\documentclass[12pt]{article}
\usepackage[utf8]{inputenc}
\usepackage[T1]{fontenc}
\usepackage{XCharter}
\title{Andrew Marvell’s Gender}
\author{Art Kavanagh}
\date{\today}

\newcommand{\citedtitle}[1]{\textit{#1}}

\begin{document}
\maketitle

\section*{Introductory note}
\textit{This essay was originally published in \textrm{Essays in Criticism}, April 2016. The final and definitive published version can be found on the journal’s site. If you wish to cite the essay, please refer to that publication. This version was generated by} pdf\LaTeX{} \textit{from the approved draft. There are minor differences in detail between it and the final, published essay.}

\section*{Essay body}
It
has been apparent for many years that Andrew Marvell often seems to identify
object with subject, or to treat them as interchangeable: to roll them up into
one self-enclosed unit which contains its own opposite. In 1978, Christopher
Ricks, contributing to the York lectures on the tercentenary of Marvell’s
death, noted that a characteristic figure in Marvell’s poetry is the use of
reflexive imagery, where something is compared to itself or to an aspect of
itself; so the drop of dew is ‘\emph{Like} its own Tear’ – what Empson calls
a ‘self-inwoven’ simile.\footnote{Christopher Ricks, ‘“Its
Own Resemblance”’, in C. A. Patrides (ed.), \citedtitle{Approaches to Marvell:
The York Tercentenary Lectures}, (1978), pp. 108–35 (p. 111). All emphasis in
quotations is original.  William Empson, \citedtitle{Seven Types of Ambiguity} (1930;
1953), 160.}
Such reflexive \emph{imagery} is merely a subset of a
more general reflexivity which pervades Marvell’s writing: for example, when
Chlora ‘courts herself in am’rous rain; / Herself both Danaë and the show’r’
(‘Mourning’, ll. 19–20), the reflexivity is not exactly a property of the
metaphor (though it is closely tied to it). In the next lecture, John Carey
says that ‘situations in which an agent finds its actions shooting back upon
itself … are curiously frequent in Marvell’s poems once one starts to look’\footnote{John Carey, ‘Reversals
Transposed: an Aspect of Marvell’s Imagination’, in Patrides (ed.), \citedtitle{Approaches
to Marvell}, pp. 136–54 (p. 144).}: ‘Round
in itself incloses’; ‘By his own scythe, the mower mown’ and the cunning with
which Cromwell induces Charles to become his own pursuer, chasing himself ‘To
Caresbrook’s narrow case’.\footnote{‘On a Drop of Dew’, l.
6; ‘Damon the Mower’, l. 80; ‘An Horatian Ode’, l. 52.}
Carey sees this reflexivity as related to something with which, he argues,
Marvell was preoccupied: ‘the self-defeating reversibility of our actions.’
Carey suggests that the point of Marvell’s use of reflexivity is to warn us
that everything we do, whatever else it might accomplish, has its first and
(for us) most important effect upon ourselves.

As
Marvell erodes the seemingly essential distinction between subject and object,
it may be that the impulse to do something similar with another grammatical
feature prompted him to engage in a more wide ranging interrogation of gender.
In recommending the younger poet to John Bradshaw, Milton vouched for the fact
that Marvell had spent ‘foure yeares abroad’, in the course of which he learned
the languages of the Netherlands, France, Italy and Spain.\footnote{Nicholas von Maltzahn, \citedtitle{An
Andrew Marvell Chronology} (Basingstoke, 2005), p. 38.} He was
already very proficient in Latin, as we know both from his poetry in that
language and from his diplomatic correspondence (which came later but for which
he was already obviously prepared). It would not be surprising if such
comparative linguistic study led him to think about gender. Three of the
languages mentioned by Milton (the exception is Dutch) manage to get by quite
well without a neuter gender. Classical Latin, in contrast, made no attempt to
do so.  The reduction of the number of genders to two is a later development;
and it seems to follow that the reduction is a simplification of a system of
categorisation that had come to appear more complex than it needed to be.

English takes a different approach. While retaining
the three genders, English tended to restrict the application of masculine and
feminine to living (and sexually differentiated) beings and to treat almost
everything inanimate as neuter.\footnote{Anne Curzan, \citedtitle{Gender
Shifts in the History of English} (Cambridge, 2003), p. 19, writes
of ‘the transformation of the English gender system’ between Old and Middle
English. ‘As grammatical gender erodes in the noun phrase in early Middle
English … the personal pronouns are the only forms to retain gender, and they
shift to natural gender. Pronominal gender systems, in general, tend to favor a
shift to semantic assignment.’}
It could be argued that this had two complementary effects. In the first place,
by appertaining to many more things than either of the others, neuter came to
seem hardly a gender at all, but rather the absence of gender.\footnote{Curzan cites P. A.
Erades, among others, as concluding that ‘English has no gender at all’:
Curzan, \citedtitle{Gender Shifts in the History of English}, p. 22. As against that, Corbett
proposes that ‘languages in which pronouns present the only evidence for gender
should be recognized as having a gender system’, distinguishing them as
‘pronominal gender systems’: Greville G. Corbett, \citedtitle{Gender} (Cambridge, 1991), p.
5.} At the
same time, the distinction between sex and gender as systems of classification
became much less significant, since it could be assumed that there was a high
degree of overlap between the female and the feminine on the one hand, and
between the masculine and the male on the other. In short, there was a tendency
for gender to appear as a simple binary system of opposites. In the case of
someone educated in Latin, this tendency was counteracted by the periodic
reminder that neuter did not govern a realm free of gender, but was just one
gender among three. In many contexts, the feminine would appear to be the
opposite of the masculine, in that to deny the latter quality to some action or
person was to ascribe to it the former. In others, however, neuter could seem
to be the negation of both masculinity and femininity, forcing them into a
temporary, unstable alliance. As a result, each gender may appear to have not
one opposite but two, which are themselves in conflict. Whether this makes,
say, masculinity seem more vulnerable (as assailed on all sides) or less (as
facing a divided opposition) is very much a matter of perception, which is
liable to shift from one moment to another, particularly if the perceiver
already fears he may be under attack. Some of the hostile responses to the
first part of \citedtitle{The
Rehearsal Transpros’d},
notably Richard Leigh’s \citedtitle{The Transproser Rehears’d}, do not find it
necessary to reconcile the imputation of effeminacy with the suggestion that
their target is a eunuch.\footnote{Paul Hammond,
‘Marvell’s Sexuality’, \citedtitle{Seventeenth Century}, 11 (1996), 87–123:
91–3.}
Presumably, this imprecision was not seen as weakening the attack.

How,
then, would somebody whose first language was English, and who had next
acquired an enviable command of Latin, respond to the discovery of French,
Spanish and Italian? If, like Marvell, he was someone whose particular habits
of mind brought him to attempt the fusion (and, failing that, the confusion) of
the agent with the acted-upon, it is to be expected that the opportunity to
shift between a binary opposition and something whose conflicts are less stark and more complex would appeal to his imagination. His poetry provides us with
evidence that this was indeed ry opposition and something whose conflicts are
less stark and more complex would the case.

The
tendency of the English approach to make neuter all but invisible and to align
the two other genders closely with the sexes is illustrated in Paul Hammond’s
discussion of Marvell’s pronouns. Having noted that ‘Young Love’ ‘includes no
gendered nouns or pronouns’, thus providing ‘indefinition’ which ‘creates space
for the reader’s mind to play’, Hammond continues:

\begin{quote}
Something
similar happens in ‘The Definition of Love’ – love which is ‘begotten by
despair / Upon Impossibility’ – where once again there are no pronouns to
indicate the sex of the poet’s desired lover.\footnote{Paul Hammond,
‘Marvell’s Pronouns’, \citedtitle{Essays in Criticism}, 53 (2003), 219–35:
221.}
\end{quote}

However,
while the ‘object’ may not have a sex, it is represented by a pronoun whose
gender is as conspicuous as it is unexpected: the neuter ‘It’:

\begin{verse}My
love is of a birth as rare\\
As ’tis for object strange and high:\\
It was begotten by Despair\\
Upon Impossibility. (‘The Definition of Love’, ll. 1–4)
\end{verse}

Love,
which is called an object, is grammatically the subject of the sentence. Since
it is ‘My’ love, the person whose object it is must be the speaker. But, when a
person speaks of ‘my love’, we expect the object to be another person, not (as
Hammond says is the case here) an abstraction. Since love is, in differing
senses, both subject and object, we seem to be close to that reflexivity which
Carey and Ricks found characteristic of Marvell.

So,
the way in which Marvell plays with gender has something in common with his
games with subject and object; and it is in ‘The Definition of Love’ that the
connection is easiest to discern. The word ‘subject’ is present in the poem
only by implication, of course, but its range of signification has implications
for Marvell’s use of ‘object’. ‘Subject’ derives from the Latin sub and iacere
(which suggest being thrown under) and applies to someone who is under the rule
of another. In that sense, a subject is necessarily the object in some
relationship, such as that of ruled to ruler, or contingent event to condition.
Yet we regularly find ourselves attempting to use it to mean something more
autonomous: a (relatively) undetermined originator of actions, as in the
subject of a sentence. This contradiction is less stark than it might seem at
first, in that the subject of any sentence is – at least potentially and more
often actually – the object of a different one: the wholly undetermined actor
is extremely rare. It is striking, too, that there we find a similar
contradiction in the term ‘agent’, which means someone who acts, but also
refers to the representative of another, an entity which can be seen as merely
an instrument of its principal.

That
two words which designate an entity which enjoys a degree of freedom and
autonomy also mean someone who is controlled or directed by someone else
suggests that we are so ill at ease with the notion of unconstrained freedom of
action that we choose to speak about it using words which tend to negate
themselves. ‘Object’ is a less obviously self-contradictory term but it too
contains a range of barely compatible meanings. It is the passive term in a
relationship, the one which suffers the effects of an action rather than being
its cause. Yet we use ‘object’ and (adjectival) ‘objective’ to refer to things
which have an existence independent of and external to ourselves:  to refer to
something as an object is often to ascribe to it a quality of
self-containedness or self-sufficiency. In its sense of \emph{aim} or \emph{goal}, it is not the thing
which acts but, perhaps more importantly, the reason the actions are taken.
Marvell glances at something of the sort in \citedtitle{The First Anniversary}:

\begin{verse}
Thee proof beyond all other force or skill\\
Our sins endanger, and shall one day kill (ll. 173–4)
\end{verse}

English
words generally do not have case markers. As with gender, pronouns are the
exceptions, though even pronouns may hide their case (compare \emph{I/me} or \emph{they/them} with \emph{you} and \emph{it}). In these lines, the
object is the very first word in the sentence, as its case makes clear. It is
followed by a subordinate clause which qualifies it, so we reach the subject
and verb only in the next line. Placing the object first and insulating it from
the other essential elements of the sentence, Marvell emphasises its
significance (in the process, drawing a parallel between the Lord Protector and
Christ). Elsewhere, of course, he relies on the absence of case markers in
English nouns to open the possibility of reading a clause in both directions.\footnote{For example, in ‘An
Horatian Ode’, lines 21–4, ‘Cæsar’s head’ is the object of the clause, but can
appear to be its subject if we read ‘blast’ as intransitive: see Barbara
Everett, ‘The Shooting of the Bears: Poetry and Politics in Andrew Marvell’, in
\citedtitle{Poets in their Time: Essays on English Poetry from Donne to Larkin} (Oxford, 1991), pp. 32–71
(p. 46).}

In
the context of love that ‘is for object strange and high’, it is clear that the
primary sense is an ‘objective’ (used as a noun); it is equally clear that this
is not the only sense in play. That the object is neuter (and therefore
impersonal) reinforces the sense that it will not be changed: the relationship
(whatever it may be) will not have a lasting or noticeable effect on it. The
actions of the subject hardly concern it, though they are all about it.

In the next line, we learn who were the parents of this love-object, with Despair
taking the male role and Impossibility the female.\footnote{Compare Cromwell as
‘the War’s and Fortune’s son’ (‘An Horatian Ode’, l. 113), where the respective
sexes of the parents are not made explicit, though they are not in doubt:
Marvell does not depart from the convention by which Fortune was depicted as
female.} Despair
is ‘magnanimous’, presumably in freeing us from the distracting and potentially
immobilising illusion of ‘feeble’ hope (l. 7). Can one still have an object if
one has truly relinquished hope of attaining it? Though a term in a
relationship, the object is not dependent on the relationship for its existence
– even for its existence \emph{as object}, in at least one sense of that
word.

We may suspect that, by assigning the pronoun ‘It’ to the subject which is also an
object, Marvell creates a kind of amphibium.\footnote{See ‘Upon Appleton
House’, l. 774; ‘The Unfortunate Lover’, l. 40; and Derek Hirst and Steven N.
Zwicker, \citedtitle{The Orphan of the Hurricane} (Oxford, 2012), pp. 81–2.}
If each of the other genders closely corresponds to one of the sexes, there
would seem to be just two possibilities open to neuter: it may represent the
exclusion of both sexes or their unity (either a eunuch or an androgyne). The
very starkness of the opposition between these two alternatives may give rise
to a third in which our abstraction hovers, balanced between the poles of
mutual exclusion and mutual comprehension. An attempt to interpret the neuter
Love as including (or excluding) both male and female is open to the objection
that, as Hammond warns us, the language of the poem does not warrant the
assumption that the object of the speaker’s love is female. Strictly speaking,
neither is there any explicit statement that the speaker is male, though we
easily infer it from (primarily) the imagery, the ‘decrees of steel’ and ‘truly
parallel’ lines (ll. 17, 27). At the very least it can be said that, when a
poet writes of love and its\footnote{Where the context so
admits, words importing one gender should be construed as including the other
two.}
object, he or she opens the \emph{possibility} that both sexes and more than one
gender are involved.

Hammond
elsewhere draws our attention to the connection in Marvell’s poetry between the
reflexive quality (or unity of subject and object) on the one hand and the
elusiveness of gender on the other. The connection is at its most obvious in
the reference to Douglas’s ‘yellow locks’, which ‘curl back themselves to seek’
(\citedtitle{The Last Instructions},
l. 653). Describing the story of Narcissus as ‘a multivalent myth’, Hammond
tells us:

\begin{quote}
Readers of Marvell’s poetry have remarked upon his recurrent use of figures of reflection, enclosure and self-resemblance. Repeatedly, Marvell imagines something seeing or seeking its own reflection, being like itself, being satisfied only with its own reflection.\footnote{‘Marvell’s Sexuality’, pp. 101–2.}
\end{quote}

Part
of the multivalency of this myth, according to Hammond, lies in its serving
simultaneously as a figure of autoeroticism or self-love and of homoeroticism,
since:

\begin{quote}
… the male gaze is enraptured by a male image, heedless of the charms of the female represented by Echo. (p. 102)
\end{quote}

Hammond
makes a convincing case that the Narcissus myth is a powerful presence in
Marvell’s poems, even though it is mentioned explicitly only once, in the
context of the river which is compared to a ‘\emph{Chrystal Mirrour} … Where all things gaze
themselves, and doubt / If they be in it or without.’\footnote{Hammond, ‘Marvell’s Sexuality’, pp. 102–3, quoting ‘Upon Appleton House’, ll. 636–8. It has been pointed out that the myth is present by implication in the final line of ‘Upon a Eunuch: a Poet’: Hirst and Zwicker, \citedtitle{The Orphan of the Hurricane}, p. 113; they also cite \citedtitle{Poems of Andrew Marvell}, ed. Nigel Smith, (2003), p. 82, n. to l. 27 of ‘To His Coy Mistress’.} The
omission of the preposition ‘at’ makes it sound as if studying one’s reflection
is a reflexive act of self-constitution. This is a figure which occurs
elsewhere in Marvell’s poetry and it may be that the association with Narcissus
is implied in those other instances as well. One example (discussed by Ricks, ‘“Its
Own Resemblance”’, p. 110) is to be found in ‘An Elegy Upon the Death of
My Lord Francis Villiers’:

\begin{verse}
Lovely and admirable as he was,\\
Yet was his sword or armour all his glass.\\
Nor in his mistress’ eyes that joy he took,\\
As in an enemy’s himself to look. (ll. 51–4)
\end{verse}

Unlike
Echo, Villiers’s mistress is not ignored but rather demoted to an inferior
position: it is in viewing himself in an enemy’s eyes that the young soldier
experiences most ‘joy’ (though there does not seem to be any hint that he neglected
Mrs. Kirke). A few lines earlier, Marvell has offered an unequivocal answer to
the question whether attributes of the opposing sexes mutually cancel each
other or are complementary:

\begin{verse}
’Tis truth that beauty doth most men dispraise:\\
Prudence and valour their esteem doth raise.\\
But he that hath already these in store,\\
Cannot be poorer sure for having more.\\
And his unimitable handsomeness\\
Made him indeed be more than man, not less. (ll. 63–7)
\end{verse}

Prudence
and valour are manly (and therefore presumably masculine) qualities, according
to this. Villiers’s manliness is not diminished by the addition of beauty, an
attribute which is, in most circumstances, inimical to masculinity: ‘beauty
doth most men dispraise’. It is not here treated as specifically \emph{feminine}, unless one assumes
that masculine and feminine form a pair of binary opposites.

The
association of possibly Narcissistic reflection, unmasculine good looks and
self-constitution in Marvell’s work is at its most apparent in \citedtitle{The
Last Instructions to a Painter}. Like Villiers, the Dutch Admiral De Ruyter seems to
bring himself into existence with his own reflected gaze:\footnote{Self-generation in Marvell is not always or only associated with reflections. I have argued that, if it is true that Marvell uses imagery of cæsarean section in  ‘An Horatian
Ode’ ll. 13–16 (Jim Swan, ‘“Cæsarean Section”: The Destruction of Enclosing Bodies in Marvell’s “Horatian Ode”’, \citedtitle{Psychocultural Review} 1 (1977), 1–8), it must
be significant that it is through Cromwell’s \emph{own} side that he divides ‘his fiery way’: Art Kavanagh, \citedtitle{Andrew Marvell’s Ambivalence about Justice}, unpublished PhD
thesis, (Royal Holloway University of London, 2012), pp. 47–8.}

\begin{verse}
The sun much brighter, and the skies more clear,\\
He finds the air and all things sweeter here.\\
The sudden change, and such a tempting sight\\
Swells his old veins with fresh blood, fresh delight.\\
Like am’rous victors he begins to shave,\\
And his new face looks in the English wave. (ll. 529–34)
\end{verse}

In
each of these cases, the absence of the preposition (‘at’) highlights the
tightness of the circle of reflexivity (and of reflection). Marvell invites us
to imagine that the object at which De Ruyter, Villiers or any of the denizens
of Nun Appleton estate looks so intently is not something separate from the
gazer – it is the gazer him- or herself.

Captain
Archibald Douglas, who was burnt alive on the \textit{Royal Oak}, exhibits apparently
feminine qualities of beauty and delicacy which, like those of Lord Francis
Villiers, complement rather than take away from his manly virtues of courage
and obedience to duty. Much has already been written, particularly in the last
two decades, about the treatment of Douglas in \citedtitle{The Last Instructions to
a Painter} but some puzzles remain. According to John Creaser:

\begin{quote}
The strangest but also the most distinctively Marvellian episode in the Restoration
satires is the death by fire of the soldier Archibald Douglas … Although
Douglas was a married man in real life, and although he is being celebrated for
his valour, he is lingeringly described as a virginal and epicene beauty, an
Adonis or Leander, with some of the qualities of Narcissus.\footnote{John Creaser, ‘“As one
scap’t strangely from Captivity”: Marvell and Existential Liberty’, in Warren
Chernaik and Martin Dzelzainis (eds.), \citedtitle{Marvell and Liberty}, (Houndmills,
Basingstoke, 1999),  pp. 144–172 (pp. 154–5).}
\end{quote}

Having
cited lines 653–4 of the poem, Creaser adds that, as Marvell misleadingly
presents it, Douglas’s ‘life has been one of pervasive but unrealized
sexuality’. In Creaser’s formulation, we have an apparent opposition, one of the
poles of which contains two elements which are not entirely at one. He
suggests, in the first place, that the valour seems to be in conflict with the
epicene quality but that in turn is undercut by the entirely fictional
virginity. If Douglas’s beauty suggests \emph{unrealized} sexuality, it is to
that extent less epicene: it is closer to asexuality than to androgyny. It
seems to follow that Marvell presents the beautiful Douglas as tending more to
the neuter than to the feminine – but unstably so. (Whatever other reasons he
may have for suppressing Douglas’s roles as father and husband, Marvell implies
that a soldier going dutifully to his death necessarily acts as someone who has
not, in Bacon’s phrase, given hostages to fortune.)

On
the face of it, whether we regard the presentation of Douglas as epicene,
androgynous or one of ‘pervasive but unrealized sexuality’, the association of
such qualities with his undoubted heroism seems to require explanation. David
Farley-Hills suggests that the explanation is to be found in the ‘inversion’ of
our expectations:

\begin{quote}
It is only in a country where standards have been inverted that the poet must seek heroic standards among his country’s enemies or foreign allies … [I]n the loyal Scot passage there is an intentional irony in the stress on Douglas’s apparent effeminacy and actual virility (649–52), which inverts the ‘female’ Stuart’s much publicised virility, but actual effeminacy. … It is no accident that the positive values of the poem inhere principally in two foreigners, a Dutchman [De Ruyter] and a Scot [Douglas]. In this the poem is an inversion of the great national epic that Renaissance writers such as Spenser sought to write to celebrate the best values of their own societies.\footnote{David Farley-Hills, \citedtitle{The Benevolence of Laughter: Comic Poetry of the Commonwealth and Restoration} (1974), pp. 85, 87–8.}
\end{quote}

As
far as it goes, this makes sense of the treatment of Douglas and De Ruyter in
the context of a satire on the conduct of the English Court. That is not to
say, however, that it fully accounts for the strangeness (as Creaser puts it)
of depicting the heroic soldier as an ‘epicene beauty’ while at the same time
drawing parallels between him and the enemy admiral.

The
treatment of both Villiers and Douglas suggests that Marvell is using the close
connection (which does not, however, amount to an exact alignment) between sex
and gender in the English natural gender classification, to show that
masculinity is not essential to maleness (nor, by implication, do femaleness
and femininity necessarily go together). To complicate the situation further,
if the ‘manly’ virtues of courage and steadfastness do not necessarily go with
masculine characteristics, neither can they be said to be inherent in maleness.
The satire provides several examples of men who have failed to exhibit these
virtues, from the Chancellor, Hyde, who balks at summoning the Parliament (ll.
469–74), to Douglas’s fellow-soldiers, who abandon their posts (ll. 629–48). It
may be precisely because being a male does not necessarily entail manliness
that Marvell is keen to show that neither does masculinity.

Marvell’s
feminine young men, whatever else they may be, are conspicuously brave. The
reader may not be able to resist raising an eyebrow at the ascription of
‘Prudence’ (l. 64) to Villiers but the nobleman’s valour is evident; just as
Douglas’s courage is not disputed, even if one agrees with Creaser that there
is an element of stubborn self-regard to his finally pointless self-sacrifice
(‘Marvell and Existential Liberty’, pp. 155–6). So far as we can judge from his
writings, Marvell placed a high value on bravery, a value that was in part a
function of its rarity: ‘We are all venal cowards, except some few’.\footnote{\citedtitle{The
Poems and Letters of Andrew Marvell}, ed. H. M. Margoliouth, 3rd edn. by Pierre Legouis with E. E. Duncan-Jones, 2 vols. (Oxford, 1971), ii. 317.} It is
worth noting that, where Marvell presents a female persona who shows masculine
characteristics, it is less clear that she has a similarly admirable overriding
quality.  While it would be difficult to deny that this differing treatment of
men and women owes something to misogyny, perhaps a more important factor is
that Marvell is careful to impute apparently feminine qualities only to these
men who have manly virtues ‘in store’, lest he appear to ‘dispraise’ them.

On
the face of it, when in \citedtitle{The Third Advice} Lady Albemarle is
represented as an animal, the sex of that animal does not seem to be of any
significance. She is the ‘monkey Duchess’ (l. 171), seen in the ‘posture just
of a four-footed beast’ (l. 186). The only indication that the monkey and the
four-footed beast are \emph{masculine} animals comes in a couplet whose
line-endings are anything but:

\begin{verse}
She dried no tears, for she was too viraginous:\\
But only snuffing her trunk cartilaginous (ll. 191–2)
\end{verse}

‘Viraginous’,
while importing masculine characteristics, is an adjective that is ordinarily
applied only to a woman. It thus illustrates in a single word the idea that
gender and sex are not precisely coterminous categories, however monstrous or
grotesque their separation might appear. Further, though \citedtitle{The
Third Advice} on its own does not permit us to draw any very firm conclusion as to the sex of
the beast to which the duchess is compared, it is tempting to read this satire
in conjunction with \citedtitle{The Last Instructions}, where her husband too is likened
to an animal. As Albemarle helplessly watches the Dutch capture the \textit{Royal
Charles},
the poet draws a parallel between the general’s frustration and rage and those
of a tigress who, from the far side of the river, sees her cubs taken by
‘Robbers’ (l. 624):

\begin{verse}
At her own Breast her useless claws does arm;\\
She tears herself since him she cannot harm. (l. 627–8)
\end{verse}

If
it is legitimate to read these two Painter poems as parts of a single narrative,
the treatment of Albemarle as a fierce, ineffectual tigress and his duchess as
an implicitly male monkey complicates our view of Lady Albemarle as a
Cassandra-like truth-teller.\footnote{See Martin Dzelzainis,
‘“Presbyterian Sibyl”: Truth-telling and Gender in Andrew Marvell’s \citedtitle{The
Third Advice to a Painter}’, in Jennifer Richards and Alison Thorne (eds.), \citedtitle{Rhetoric,
Women and Politics in Early Modern England}, (2007), pp. 111–28.}

It
would appear from the foregoing that any discussion of gender in Marvell will
have difficulty in getting very far away from \citedtitle{The Last Instructions}. This satire also
contains Marvell’s allusion to the Skimmington Ride, which he describes as

\begin{verse}
A punishment invented first to awe\\
Masculine wives transgressing Nature’s law (ll. 377–8)
\end{verse}

The
‘brawny’ husband-beater is not the only one subjected to the punishment. She
and the husband who has failed to control her are alike mounted on ‘lean jade’,
a worn-out horse, and paraded through the streets to the ‘hooting’ of children
and the banging of sticks on kettles. Marvell presents this as a form of
community justice, preferable to a jury’s award of damages to the husband or
the binding of the wife by ‘partial justice’ to keep the peace:

\begin{verse}
Prudent Antiquity, that knew by shame,\\
Better than law, domestic crimes to tame (ll. 387–8)
\end{verse}

He
encourages the painter to whom the satire is addressed to join him in likewise
subjecting offending behaviour to ridicule:

\begin{verse}
So thou and I, dear painter, represent\\
In quick effigy, others’ faults, and feign\\
By making them ridiculous, to restrain (ll. 390–2)
\end{verse}

The
use of ‘feign’, given emphasis by its position as a line-ending rhyme word, is
the first clear hint we are given that the poet is less approving of the
methods of ‘Prudent Antiquity’ than he says he is. Nigel Smith suggests that
two of the \citedtitle{OED}’s definitions,
‘contrive’ and ‘pretend’, are at work.\footnote{\citedtitle{Poems of Andrew Marvell}, p. 378, note to l. 391.}
Marvell has used ‘feign’ similarly as a rhyme word in ‘Tom May’s Death’ (ll.
95–6), a poem centrally concerned with fiction, and in particular the fiction
of a punishment which fits the crime: it features a poet accused of writing
misleading history, sentenced in Elysium by the shade of another poet to ‘what
torments poets ere did feign’.\footnote{I develop this argument
in Kavanagh, ‘Andrew Marvell’s Ambivalence about Justice’, pp. 68–9.}
It is possible that, in \citedtitle{The Last Instructions}, ‘feign’ is used in a
third sense as a homophone: the poet and the painter would \emph{fain}
correct the faults of
the court and the king’s ministers by holding them up to ridicule but they have
no great hopes that their attempts to rectify the actions of those in power
will succeed. Because justice demands at least that the absurdity of official
actions should be ridiculed, they will pretend to believe that they are
performing this function, even if the exposure does not work to ‘restrain’ the
offending behaviour.

Marvell
appears to draw the lesson of the Skimmington Ride by spelling out the
parallels between the disputing neighbours and the sovereign powers of Europe:

\begin{verse}
So Holland with us had the mastery tried,\\
And our next neighbours, France and Flanders, ride (ll. 395–6)
\end{verse}

According
to Smith, in this analogy ‘Holland is the masterful wife, England the beaten
husband, France and Flanders the neighbours.\footnote{\citedtitle{Poems of Andrew Marvell}, p. 378, note to l. 396.}
However, for the analogy to hold, justice would require that France and
Flanders have no more interest in the dispute than the wish to uphold the
community’s norms. In fact, each of them is more directly implicated in the
power-play. Jermyn, in his negotiations with France, will shortly be ordered
‘To play for Flanders and the stake to lose’ (l. 368). I have argued elsewhere
that, while the Dutch are fairly portrayed in \citedtitle{The Last Instructions} as having acted
perfidiously, the behaviour of the French is greatly more culpable and the
English court tends to copy France more closely than it does the Netherlands
when it comes to bad faith. The Dutch, while certainly not blameless, are the
nearest thing to an innocent party in the proceedings.\footnote{Kavanagh, ‘Andrew Marvell’s Ambivalence about Justice’, p. 111.}
Certainly, France cannot claim to be impartial.

Line 396 is not easy to reconcile with an
interpretation that sees the international political situation as conforming to
the pattern of the Skimmington Ride. According to the description that Marvell
has just given of the punishment, it is the couple who fail to observe
established gender roles who are supposed to ‘ride’ on the decrepit horse. It
is conceivable that he is using an implied ellipsis to suggest that ‘our next
neighbours, France and Flanders, [force us to] ride’ but given the less than
impartial role of those ‘next neighbours’ it would probably be a mistake to
attempt to explain away the apparent confusion as to the nature of their
participation. And the confusion is compounded when St Albans is instructed to
complain to Louis

\begin{verse}
How yet the Hollanders do make a noise,\\
Threaten to beat us, and are naughty boys (ll. 429–30)
\end{verse}

The
threats to beat ‘us’ are clearly consistent with casting the Dutch in the role
of a masculine wife; however the description of them as naughty and noisy boys
seems to place them equally in the position of jeering neighbours. In the
international version of the Skimmington Ride, the powers of Europe seem to
have trouble keeping to their assigned parts.

Such
confusion as to their proper roles, combined with the evident self-interest of
some of the parties imposing the punishment, warn us that we should be wary of
taking entirely at face value Marvell’s characterisation of the practice as
‘Better than law’. Indeed, lines 377–8 may afford us another example of
Marvell’s slippery syntax: on the face of it, it is the wives’ masculinity that
constitutes the transgression of nature’s law, but might it not equally be the invention
of the punishment? There is evidence that Marvell was suspicious of informal or
untried procedures for dealing with alleged wrongdoing, where the risk of a
breach of natural justice is particularly great. He was critical of the Commons
for proceeding ‘Summarily within themselves’ against Buckingham instead of
impeaching him,\footnote{\citedtitle{Prose
Works of Andrew Marvell}, ed. Annabel Patterson and others, 2 vols. (New Haven, 2004), ii. 275.}
partly because impeachment would have meant a hearing before a different body
(the Lords) from the one making the complaint (the Commons). His opposition to
the impeachment of Clarendon, whom he excoriated in the \citedtitle{Painter} poems, has not been
fully explained but it is telling that he drew a parallel between Clarendon’s
case and Buckingham’s, while warning against any ‘sudden’ move by the Commons
against the former.\footnote{Nigel Smith, \citedtitle{Andrew
Marvell: The Chameleon} (New Haven, 2010), p. 208; von Maltzahn, \citedtitle{An Andrew Marvell
Chronology}, pp. 99–100 and Hirst and Zwicker, \citedtitle{The Orphan of the Hurricane}, p. 155.}
He further condemned the punishment of Buckingham, Shaftesbury, Wharton and
Salisbury as an unprecedented ‘Imprisonment without Example’ (\citedtitle{Prose
Works}, ii. 297).

The
‘female Stuart’ motif noted by Farley-Hills is similarly present in Marvell’s
treatment of the capture of the \textit{Royal Charles}. He makes full use of
the facts that the pronouns traditionally applied to ships are feminine and
that the Dutch prize bears the same name as the monarch:

\begin{verse}
The pleasing sight he [De Ruyter] often does prolong:\\
Her masts erect, tough cordage, timbers strong,\\
…\\
The seamen search her all within, without:\\
Viewing her strength, they yet their conquest doubt;\\
Then with rude shouts, secure, the air they vex,\\
With gamesome joy insulting on her decks (ll. 727–8, 731–4)
\end{verse}

The
gender elision in this passage could hardly be more complete. The vessel with
the king’s name has properties which are ‘erect’, ‘tough’ and ‘strong’. Despite
‘her strength’, she is conquered and searched ‘within’ as well as without.\footnote{See Barbara Riebling,
‘England Deflowered and Unmanned: The Sexual Image of Politics in Marvell’s
“Last Instructions”’, \citedtitle{SEL: Studies in English Literature
1500–1900 }, 35 (1995), 137–57: 149.} The
anguish exhibited in the implied comparison of the national humiliation with a
sexual assault is tempered by a sense that the humiliation of king and his
advisers is even greater – and there is a clear implication that they have been
asking for it.

It
may be that the focus adopted by Hirst and Zwicker, who write about Marvell’s
being caught up in ‘the toils of patriarchy’, could usefully be narrowed to
concentrate on the toils, more specifically, of \emph{masculinity}. According to Hirst and
Zwicker:

\begin{quote}
The enduring political circumstance within which Marvell was situated was …
patriarchy; his enduring social circumstance – orphan and isolate, tutor and
landless politician – was dependency. His enduring condition as a writer was at
once to yearn for the shelter, and to feel the oppressions, of patriarchy and dependency
alike.\footnote{Derek Hirst and Steven Zwicker, ‘Andrew Marvell and the Toils of Patriarchy: Fatherhood, Longing and
the Body Politic’, \citedtitle{ELH: English Literary History} 66 (1999), 629–54: 631.}
\end{quote}

If
patriarchy affords \emph{oppressive shelter}
to those under its ‘cruel care’ (‘The Unfortunate Lover’, l. 29), an analogous
claim might be made about masculinity: it is a quality that can appear as
simultaneously a protection and a threat. It offers to protect us by announcing
to the world at large that we are strong, brave, willing to defend ourselves
when necessary and that what we may lack in combat skills will be made up in
ferocity and determination.
On the other hand, this very proclamation of our preparedness to fight may lead
us into situations where we shall be required to prove it, particularly as we
are likely to find that our vaunted masculinity brings us into association with
other men who are just as busily vaunting theirs. Hirst and Zwicker do not have that
quite in mind: for them, the oppression that comes with patriarchy in no way
lessens the shelter it provides; rather, it can be seen as the high price one
pays, in much the same way as the Hobbesian social contractor signs away
autonomy in return for the prospect of security. The bargain involved in
masculinity is more akin to a gamble: we may accept the odds that we shall be
safer if we adopt attitudes and characteristics of strength and aggression but,
unless we are fools, we are aware of the risk that the bet will go the other
way. When it does, masculinity becomes an example of Carey’s ‘self-defeating
reversibility’. ‘Force does not establish power; it establishes, simply, the
need to use force’, he tells us.\footnote{Carey, ‘Reversals Transposed’, p. 147.}
In much the same way, chest-thumping does not necessarily keep us safe; it may
merely mean that we need to thump our chests even harder, if we wish to avoid
having someone else do it for us.\footnote{This does not, of
course, purport to be a comprehensive account of masculinity. See Diane
Purkiss, ‘Thinking of Gender’, in Derek Hirst and Steven N. Zwicker (eds.), \citedtitle{The
Cambridge Companion to Andrew Marvell}, (Cambridge, 2011), pp. 68–86 (pp.
80–1) for another reason why a Renaissance writer might have been uncomfortable
in a masculine role.}

It
would not be surprising, then, if masculinity had appeared to Marvell, as it
appears to many men, in a dual character, at once reassuring and repellant. A
case can be made that femininity likewise appears to be double and
self-contradictory, though not for precisely corresponding reasons. In ‘The
Gallery’, it may well be retorted, femininity shows itself not so much dual as
multiform. It remains true, however, that the successive portrayals fall
alternately into one of two broader categories: the alluring and the
threatening. Although the speaker introduces only one image at a time, the
inevitability of the passage to the next means that a threat always lies behind
the allure. The final stanza’s portrait seems relatively artless and less
affected than the others but it is no more authentic on that account.
Furthermore, though it is the last to be shown, it comes first in time: it
represents ‘the same posture, and the look … with which I first was took’ (ll.
51–2). Chronologically, attraction comes before danger; in terms of importance
the precedence is reversed.

It
is therefore tempting to conclude that the latter is the reality, the former
the disguise, and that the lyric should be read as a warning against the
concealed trap of female attractiveness. As against this, \emph{each} of the various
portraits which comprise the gallery is a ‘posture’; and the postures are so
extreme that they can only be exaggerations: witch or homicidal torturer on the
one hand, goddess on the other. Clora has never actually raved over her lover’s
entrails in a cave, any more than she has floated on the sea in a huge shell.
Though less extreme, even the ‘tender shepherdess’, a staple of pastoral, is an
obviously fictional figure. None of the pictures is a close representation of
reality, which is surely something quite different and may be unknowably so. It
is not a long step from this to the recognition that femininity (in its diverse
forms) is itself a series of assumed postures – nor from that to the idea that
the same might be true of masculinity.

Marvell
again engages in the detachment of a masculine role from a male and of a
feminine one from a female in ‘Clorinda and Damon’. The \textit{carpe
diem} motif is usually encountered in the univocal utterance of a male persona, the
woman’s voice remaining unheard. ‘Clorinda and Damon’ (note that the young
woman’s name comes first in the title, in contrast to ‘Ametas and Thestylis’,
‘Daphnis and Chloe’ or ‘Thyrsis and Dorinda’) is a dialogue in which it is
Clorinda who urges her companion to ‘Seize the short joys then, ere they vade’
(l. 8). Because only one voice speaks in Herrick’s ‘Gather Ye Rosebuds’ or,
indeed, in ‘To His Coy Mistress’, the reader is left with no clear sense of the
effect which the attempt at persuasion has had on its addressee. Success or
failure, this might imply, is not really the point. In Marvell’s dialogue, on
the other hand, we do learn the result: Clorinda, far from having things her
way, is instead won over to Damon’s point of view.

One
remarkable feature of Damon’s assumption of the more usually feminine role is
that it is accomplished without a sense of absurdity or any obvious element of
travesty. The Moncks are a formidable couple who are undoubtedly to be taken
seriously but their presentation (in two distinct but thematically related
poems) as a ferocious but powerless male tigress and a truth-speaking, immodest
female monkey must hold them up to some degree of ridicule, even if the
ridicule is tempered by the urgency of the predicaments in which they are
presented to us. Damon, in contrast, does not appear ridiculous: an explanation
of his unexpected behaviour is provided by its narrative context:

\begin{verse}
D. These once had been enticing things,\\
Clorinda, pastures, caves, and springs.\\
C. And what late change? D. The other day\\
Pan met me. (ll. 17–20)
\end{verse}

Clorinda
quickly changes her approach. Initially, her focus is still on Damon –‘Sweet
must Pan sound in Damon’s note’ (l. 24) – suggesting that she still has hopes
of seduction but is sharp enough to recognise that a direct assault on the
convert’s new beliefs will probably not succeed. No doubt many male wooers,
confronted with an ostensibly unshakeable ‘virtue’ grounded in religious
fervour, have adopted a similar strategy. Though Clorinda joins Damon in
singing Pan’s praises, it is not clear that in doing so she relinquishes her
masculine role.

It
is certainly arguable that the effect of this exercise in role reversal was to
reinforce patriarchal order rather than to explore its weaknesses. It is quite
possible that, in the poet’s eyes, the young shepherd who leads his companion
towards the new religion is fulfilling the proper role of his sex, even as he
avoids the corresponding gender position. But if this is true, it remains the
case that, in ‘Clorinda and Damon’, Marvell has shown us the partial separation
of the masculine and the feminine from male and female respectively, and done
so in a way that does not expose his protagonists to ridicule. Whatever may be
said about the ‘Masculine wives’ of the Skimmington Ride, neither the shepherd
nor his companion is ‘transgressing Nature’s law’.

Nor
can ‘the laws’ (‘Daphnis and Chloe’, l. 107) according to which two would-be
lovers attempt to order their behaviour be said to be those of nature.
Daphnis’s ‘manly stubbornness’ (l. 70) is revealed to be a formal role he
adopts for only so long as it does not interfere with his usual pastimes
(stanza XXVI). According to the narrator, Chloe
has been thwarted by ‘Nature, her own sex’s foe’ (l. 5) but few readers will
think her worse off for having missed her fleeting opportunity with Daphnis.
Although this poem’s final line is a question about the reason for Chloe’s
behaviour, its central puzzle concerns Daphnis’s. In this, it forms a pair with
‘Mourning’, which treats Chlora’s tears of (‘supposed’, l. 36 – but not
necessarily or always simulated) grief as a phenomenon to be carefully and
subtly interpreted. While one poem anatomises a rigid masculine insistence on
rules of behaviour which, though strict, have nothing to do with morality, the
other expatiates on the unfathomable nature of feminine weeping.

Onias IV was a Jewish high priest who was
forced to flee to Alexandria to avoid having his skull crushed by his fellow
priests. He had been tricked by his older brother, Shimei, into believing that
a gown and woman’s girdle were the priestly vestments that he was supposed to
wear. When he officiated in these garments, his brother told the other priests
that Onias had promised his beloved: ‘On the day in which I will assume the
office of high priest, I will put on your gown and gird myself with your
girdle.’ In \citedtitle{The Rehearsal Transpros’d: The Second Part}, Marvell compares Samuel Parker to
Onias, to the former’s disadvantage. Marvell accuses Parker of employing in his
works a language which indicates that he has 

\begin{quote}
forgot not only all Scripture rules, but even all Scripture expressions, unless where he either distorts them to his own interpretation, or attempts to make them ridiculous to others. Insomuch that, of all the Books that ever I read, I must needs say I never saw a Divine guilty of such ribaldry and prophaneness. (\citedtitle{Prose Works}, i. 245–6)
\end{quote}

The
point of the comparison is that Parker has clothed his arguments in a language
which is appropriate to their, as Marvell sees it, irreligious nature but which
is entirely out of place in the discourse of a divine. Parker’s work is ‘so
uncanonical and impious, that it would bear an higher and more deserved
accusation than that of Onias … for officiating in a Womans Zone instead of the
Priestly girdle, and for the sacred Pectoral wearing his Mistresses Stomacher’
(\citedtitle{Prose
Works}, i. 246). Marvell chose this obscure analogy in the context of his argument with
Parker about zealotry. He had objected to Parker’s claim that, in clearing the
moneylenders from the temple, Jesus had acted as a zealot. Jason Rosenblatt has
shown that, in this argument, Marvell had contrived to appear more erudite than
he was.\footnote{Jason P. Rosenblatt, \citedtitle{Renaissance England’s Chief Rabbi: John Selden} (Oxford, 2006), Chapter 5, especially at pages 127–8.}
Both parties relied on the work of John Selden, but Rosenblatt makes it clear
that Parker was more familiar with it than was Marvell. The story of Onias is
to be found in Babylonian Talmud Tractate \citedtitle{Menachot} 109b. However,
according to Rosenblatt (\citedtitle{Renaissance England’s Chief Rabbi}, p. 128), Marvell ‘has
taken the story without acknowledgement not from the Talmud but from Selden’s \citedtitle{De
Successione in Pontificatum Ebraeorum} (1636)’.

Whatever
the merits of Marvell’s argument about zealotry, he clearly presents a striking
image of a priest carrying out his sacred duties in feminine dress. So
effective is the image that the reader is forced to remind him- or herself that
Parker is being taxed with clothing merely his public pronouncements – not his
person – in a blasphemously impermissible garb. That Marvell should choose to
present his readership with such an image is worth remarking, even if any
explanation that might be canvassed must necessary be conjectural. Discussing
another passage from \citedtitle{The Rehearsal Transpros’d}, in which Marvell has
referred to Parker’s forcing nonconformists to ‘run the \emph{Ganteloop}’ Hammond writes:

\begin{quote}
What starts as a purely symbolic subjection to the verger’s wand of office has by the
end of the sentence become a literal (and explicitly visualized) physical punishment. … to Leigh this flight of fancy seemed evidence of Marvell’s personal predilections …\footnote{‘Marvell’s Sexuality’, p. 93.}
\end{quote}

Something
similar could be said of the image of Parker (figuratively) clad in his
mistress’s stomacher. (At least, we must assume that the stomacher was not an
actual garment worn by the archdeacon. One conceivable explanation among
several is that Marvell was engaging in a form of innuendo in which he put
forward a literal accusation in the guise of a metaphorical one. His references
to Clarendon’s ‘rupture’ (\citedtitle{Second Advice}, ll. 117–8 and \citedtitle{Last
Instructions},
ll. 473–4), seem to disguise the literal as the figurative. If he is doing
something analogous in this instance, the motive might be to deter Parker from
an escalation of hostilities.) At any rate, it is significant that Marvell was
again prepared to risk inviting the inference that his attack on Parker was
actually a betrayal of his own predilections even after Leigh had turned the
Ganteloop passage against him.

Marvell’s
imagination led him to puzzle over and test the categories of masculine and
feminine (both in comparison and contradistinction to the neuter) considering
the possibility that they might both be ‘postures’ adopted to protect (in their
different ways) the self from threat and danger. If the price of that
protection is the submission to certain constraints on one’s behaviour, we
should not be justified in concluding on that account that the postures are
‘merely’ social constructs, falling on the inessential side of a postulated
nature-nurture divide. If the term ‘natural’ has any proper application, it
must cover the strategies we adopt to shield ourselves from those risks to
which we believe ourselves susceptible.\footnote{Richard Tuck, \citedtitle{Natural Rights Theories: Their Origin and Development} (Cambridge, 1979).}
To say so is not to assert that the posture best calculated to offer protection
will be the same in all circumstances or to deny that it is liable to show
signs of instability. Facing death, neither Douglas nor Villiers could be
afforded any more protection by a show of masculinity than each derived from
his very obvious courage.

\end{document}

